\newpage

\section*{Programming Principles}
Die Softwareentwicklung dreht sich um zwei zentrale Größen, welche maßgeblich den Erfolg definieren: Zum einen sollte die Software einen Nutzen bieten, zum anderen sollte sie aber auch möglichst wenig Kosten udn Folgekosten verursachen.
Zur Erreichung der zweiten Größe, einer preiswerten Software, müssen mehrere Eigenschaften erfüllt werden: Beispielsweise sollte sie robust und benutzbar, effizient, wartbar, kompatibel und erweiterbar sein. \\
\noindent Programmierprinzipien sind Leitlinien und Grundlagen für Entscheidungen, um eben diese Eigenschaften zu erfüllen. Im Folgenden werden mehrere Bündel aus Prinzipien vorgestellt, und es soll dargestellt werden, ob und wie weit sie im vorliegenden Programmcode umgesetzt wurden.

\subsection*{SOLID}
Die SOLID-Programmierprinzipien wurden erstmalig 2000 von Robert C. Martin veröffentlicht. Sie bestehen aus 5 Regeln, SOLID stellt das Akronym der Einzelregeln dar.
\paragraph*{Single Responsibility Principle (SRP)}
"The Single Responsibility Principle states that a class or module should have one, and only one, reason to change." \cite[138]{Martin2008}
Mit diesem einen Satz fasst Martin das SRP zusammen, welches seiner Meinung nach eines der wichtigsten Konzepte der objektorientierten Programmierung darstellt.
...
\paragraph*{Open-Closed Principle}

"You should be able to extend the behavior of a system without having to modify that system."

"Plugin systems are the ultimate consummation, the apotheosis, of the Open-Closed Principle. They are proof positive that open-closed systems are possible, useful, and immensely powerful."
\paragraph*{Liskov Substitution Principle}
\paragraph*{Interface Segregation Principle}
\paragraph*{Interface Segregation Principle}



\subsection*{GRASP}


\subsection*{DRY}
