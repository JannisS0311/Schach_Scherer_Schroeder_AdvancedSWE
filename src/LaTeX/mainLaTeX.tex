%&bericht

%%%%%%%%%%%%%%%%%%%%%%%%%%%%%%%%%%%%%%%%%%%%%%%%%%%%%%%%%%%%%%%%%%%%%%%%%%%%%%%
%% Descr:       Vorlage für Berichte der DHBW-Karlsruhe
%% Author:      Prof. Dr. Jürgen Vollmer, juergen.vollmer@dhbw-karlsruhe.de
%% $Id: bericht.tex,v 1.25 2020/03/13 15:07:45 vollmer Exp $
%%  -*- coding: utf-8 -*-
%%%%%%%%%%%%%%%%%%%%%%%%%%%%%%%%%%%%%%%%%%%%%%%%%%%%%%%%%%%%%%%%%%%%%%%%%%%%%%%

\documentclass[
    ngerman          % neue deutsche Rechtschreibung
,a4paper          % Papiergrösse
% ,twoside          % Zweiseitiger Druck (rechts/links)
% ,10pt             % Schriftgrösse
%  ,11pt
,12pt
,pdftex
%  ,disable         % Todo-Markierungen auschalten
]{report}

% Bitte die Codierung Ihrer Dateien auswählen:
% \usepackage[latin1]{inputenc}    % Für UNIX mit ISO-LATIN-codierten Dateien
% \usepackage[applemac]{inputenc}  % Für Apple Mac
% \usepackage[ansinew]{inputenc}   % Für Microsoft Windows
\usepackage[utf8]{inputenc}        % UTF-8 codierte Dateien
% Dieses Dokument ist unter Unix erstellt, daher
% wird diese Input-Codierung benutzt.

\usepackage{bericht}
\usepackage{graphicx} %Bilder
\usepackage{wrapfig} %Bilder
\usepackage{subfig} %Bilder
\usepackage[bindingoffset=1cm]{geometry} %Rand für die Bindung
\usepackage{setspace} %Zeilenabstand
\usepackage{acronym} %Abkürzungen
\usepackage{biblatex}
\usepackage{csquotes}
%\usepackage{apacite}
\usepackage{paralist}
\usepackage{longtable}
\usepackage{booktabs}
%\usepackage[center]{caption}
\usepackage{float}
\usepackage{pgf-pie}
\usepackage{pdfpages}
\usepackage{setspace}
\usepackage{biblatex}


% Hierdurch werden Schusterjungen und Hurenkinder vermieden, d.h. einzelne Wörter
% auf der nächsten Seite oder in einer einzigen Zeile.
% LaTeX kann diese dennoch erzeugen, falls das Layout ansonsten nicht umsetzbar ist.
% Diese Werte sind aber gute Startwerte.
\widowpenalty10000
\clubpenalty10000


%%%%%%% Für Quellcodevorlage
\usepackage{scrhack}                    % Hack zur Verw. von listings in KOMA-Script
\usepackage{listings}                   % Darstellung von Quellcode
\usepackage{xcolor}                     % Einfache Verwendung von Farben
\input{quellcodeStyle}  % Weitere Details sind ausgelagert

\usepackage{algorithm}                  % Für Algorithmen-Umgebung (ähnlich wie lstlistings Umgebung)
\usepackage{algpseudocode}              % Für Pseudocode. Füge "[noend]" hinzu, wenn du kein "endif",
% etc. haben willst.

\makeatletter                           % Sorgt dafür, dass man @ in Namen verwenden kann.
% Ansonsten gibt es in der nächsten Zeile einen Compilefehler.
\renewcommand{\ALG@name}{Algorithmus}   % Umbenennen von "Algorithm" im Header der Listings.
\makeatother                            % Zeichen wieder zurücksetzen
\renewcommand{\lstlistingname}{Listing} % Erlaubt das Umbenennen von "Listing" in anderen Titel.

\usepackage{amssymb}    % Lädt amsfonts und weitere Symbole
\usepackage{MnSymbol}   % Für Symbole, die in amssymb nicht enthalten sind.

\csname endofdump\endcsname


%%%%%%%%%%%%%%%%%%%%%%%%%%%%%%%%%%%%%%%%%%%%%%%%%%%%%%%%%%%%%%%%%%%%%%%%%%%%%%%
%% Angaben zur Arbeit
%%%%%%%%%%%%%%%%%%%%%%%%%%%%%%%%%%%%%%%%%%%%%%%%%%%%%%%%%%%%%%%%%%%%%%%%%%%%%%%

\newcommand{\Autor}{Jannis Schröder, Annalena Scherer}
\newcommand{\MatrikelNummer}{ , 7496432}
\newcommand{\Kursbezeichnung}{TINF19B1}

\newcommand{\FirmenName}{Boehringer Ingelheim}
\newcommand{\FirmenEndung}{Pharma GmbH \& Co K.G.}
\newcommand{\FirmenStadt}{88400 Biberach an der Riß}
\newcommand{\FirmenLogoDeckblatt}{\includegraphics[width=6cm]{Images/BI_Logo.png}}

% Falls es kein Firmenlogo gibt:
%  \newcommand{\FirmenLogoDeckblatt}{}

%\newcommand{\BetreuerFirma}{Betreuer}
\newcommand{\BetreuerDHBW}{Daniel Lindner}

%%%%%%%%%%%%%%%%%%%%%%%%%%%%%%%%%%%%%%%%%%%%%%%%%%%%%%%%%%%%%%%%%%%%%%%%%%%%%%%%%%%%%

% Wird auf dem Deckblatt und in der Erklärung benutzt:
%\newcommand{\Was}{Projekt-/Studien-/Bachleorarbeit}
%\newcommand{\Was}{Projektarbeit}
%\newcommand{\Was}{Studienarbeit}
%\newcommand{\Was}{Bachleorarbeit}
\newcommand{\Was}{Advanced SWE - Programmentwurf}

%%%%%%%%%%%%%%%%%%%%%%%%%%%%%%%%%%%%%%%%%%%%%%%%%%%%%%%%%%%%%%%%%%%%%%%%%%%%%%%%%%%%%

\newcommand{\Titel}{Schachspiel}
%\newcommand{\Untertitel}{im Rahmen der AdA-Prüfung}
\newcommand{\AbgabeDatum}{15. Mai 2021}

\newcommand{\Dauer}{25 Wochen}

% \newcommand{\Abschluss}{Bachelor of Engineering}
\newcommand{\Abschluss}{Bachelor of Science}

\newcommand{\Studiengang}{Informatik / Medizinische Informatik}
% \newcommand{\Studiengang}{Informatik / Angewandte Informatik}

\hypersetup{%%
    pdfauthor={\Autor},
    pdftitle={\Titel},
    pdfsubject={\Was}
}

%%%%%%%%%%%%%%%%%%%%%%%%%%%%%%%%%%%%%%%%%%%%%%%%%%%%%%%%%%%%%%%%%%%%%%%%%%%%%%%

% Wenn \includeonly{..} benutzt wird, werden nur diese Kaptitel ausgegeben.

%%%%%%%%%%%%%%%%%%%%%%%%%%%%%%%%%%%%%%%%%%%%%%%%%%%%%%%%%%%%%%%%%%%%%%%%%%%%%%%

% Benutzt man das "biblatex"-Paket, dann muß das hier stehen:
% siehe auch die mit BIBLATEX markierten Zeilen in bericht.sty
\bibliography{bericht}

\begin{document}
%%%%%%%%%%%%%%%%%%%%%%%%%%%%%%%%%%%%%%%%%%%%%%%%%%%%%%%%%%%%%%%%%%%%%%%%%%%%%%%

    \begin{titlepage}
        \begin{center}
            \vspace*{-2cm}
            \FirmenLogoDeckblatt\hfill\includegraphics[width=4cm]{Images/dhbw-logo.png}\\[2cm]
            \vspace*{1cm}
            {\huge \Titel}\\[0.5cm]
            %{\huge \Untertitel}\\[1cm]
            {\Huge\scshape \Was}\\[1cm]
            {\large für die Prüfung zum}\\[0.5cm]
            {\Large \Abschluss}\\[0.5cm]
            {\large des Studienganges \Studiengang}\\[0.5cm]
            {\large an der}\\[0.5cm]
            {\large Dualen Hochschule Baden-Württemberg Karlsruhe}\\[0.5cm]
            {\large von}\\[0.5cm]
            {\large\bfseries \Autor}\\[1cm]
            {\large Abgabedatum \AbgabeDatum}
            \vfill
        \end{center}
        \begin{tabular}{l@{\hspace{2cm}}l}
            %Bearbeitungszeitraum	         & \Dauer 			\\
            Matrikelnummer                 & \MatrikelNummer  \\
            Kurs                         & \Kursbezeichnung \\
            Ausbildungsfirma               & \FirmenName     \\
            & \FirmenEndung      \\
            & \FirmenStadt    \\
            %Betreuer der Ausbildungsfirma	 & \BetreuerFirma		\\
            Gutachter der Studienakademie  & \BetreuerDHBW    \\
        \end{tabular}
    \end{titlepage}

%%%%%%%%%%%%%%%%%%%%%%%%%%%%%%%%%%%%%%%%%%%%%%%%%%%%%%%%%%%%%%%%%%%%%%%%%%%%%%%



    \onehalfspacing

    
\newpage
\thispagestyle{empty}
%\begin{framed}
\begin{center}
\Large\bfseries Erklärung
\end{center}
\medskip
\noindent

\noindent Ich versichere hiermit, dass ich meine \Was\ mit dem Thema: \glqq \Titel \grqq\
selbstständig verfasst und keine anderen als die angegebenen Quellen und Hilfsmittel benutzt habe. Ich versichere zudem, dass die eingereichte elektronische Fassung mit der gedruckten Fassung übereinstimmt.
\vspace{3cm}
\noindent \\
\underline{\hspace{4cm}}\hfill\underline{\hspace{6cm}}\\
Ort~~~~~Datum\hfill Unterschrift\hspace{4cm}
%\end{framed}


\vfill
%\begin{framed}
\begin{center}
\Large\bfseries Sperrvermerk
\end{center}
\medskip
\noindent
Der Inhalt dieser Arbeit darf weder als Ganzes noch in Auszügen Personen
außerhalb des Prüfungsprozesses und des Evaluationsverfahrens zugänglich gemacht
werden, sofern keine anderslautende Genehmigung vom Dualen Partner vorliegt.
%\end{framed}

\endinput

    \input{OtherInput/Abstract}
    \input{OtherInput/Zusammenfassung}

    \newpage
    \tableofcontents           % Inhaltsverzeichnis hier ausgeben
    %\listoffigures             % Liste der Abbildungen
    %\listoftables              % Liste der Tabellen
    %\lstlistoflistings         % Liste der Listings
    %\listofequations           % Liste der Formeln

    \input{OtherInput/Abkürzungen}

    \bigskip

    \section*{Anmerkung zu dieser Arbeit}
    In der vorliegenden Arbeit wird aus Gründen der besseren Lesbarkeit vornehmlich das generische Maskulin verwendet, welches männliche sowie weibliche Personen gleichermaßen einschließt. Ist beispielsweise von Mitarbeitern die Rede, werden damit auch Mitarbeiterinnen angesprochen.\\
    
\newpage
\thispagestyle{empty}
%\begin{framed}
\begin{center}
\Large\bfseries Erklärung
\end{center}
\medskip
\noindent

\noindent Ich versichere hiermit, dass ich meine \Was\ mit dem Thema: \glqq \Titel \grqq\
selbstständig verfasst und keine anderen als die angegebenen Quellen und Hilfsmittel benutzt habe. Ich versichere zudem, dass die eingereichte elektronische Fassung mit der gedruckten Fassung übereinstimmt.
\vspace{3cm}
\noindent \\
\underline{\hspace{4cm}}\hfill\underline{\hspace{6cm}}\\
Ort~~~~~Datum\hfill Unterschrift\hspace{4cm}
%\end{framed}


\vfill
%\begin{framed}
\begin{center}
\Large\bfseries Sperrvermerk
\end{center}
\medskip
\noindent
Der Inhalt dieser Arbeit darf weder als Ganzes noch in Auszügen Personen
außerhalb des Prüfungsprozesses und des Evaluationsverfahrens zugänglich gemacht
werden, sofern keine anderslautende Genehmigung vom Dualen Partner vorliegt.
%\end{framed}

\endinput


    % Jetzt kommt der "eigentliche" Text
    \newpage

\section*{Programming Principles}
Die Softwareentwicklung dreht sich um zwei zentrale Größen, welche maßgeblich den Erfolg definieren: Zum einen sollte die Software einen Nutzen bieten, zum anderen sollte sie aber auch möglichst wenig Kosten udn Folgekosten verursachen.
Zur Erreichung der zweiten Größe, einer preiswerten Software, müssen mehrere Eigenschaften erfüllt werden: Beispielsweise sollte sie robust und benutzbar, effizient, wartbar, kompatibel und erweiterbar sein. \\
\noindent Programmierprinzipien sind Leitlinien und Grundlagen für Entscheidungen, um eben diese Eigenschaften zu erfüllen. Im Folgenden werden mehrere Bündel aus Prinzipien vorgestellt, und es soll dargestellt werden, ob und wie weit sie im vorliegenden Programmcode umgesetzt wurden.

\subsection*{SOLID}
Die SOLID-Programmierprinzipien wurden erstmalig 2000 von Robert C. Martin veröffentlicht. Sie bestehen aus 5 Regeln, SOLID stellt das Akronym der Einzelregeln dar.
\paragraph*{Single Responsibility Principle (SRP)}
"The Single Responsibility Principle states that a class or module should have one, and only one, reason to change." \cite[138]{Martin2008}
Mit diesem einen Satz fasst Martin das SRP zusammen, welches seiner Meinung nach eines der wichtigsten Konzepte der objektorientierten Programmierung darstellt.
...
\paragraph*{Open-Closed Principle}

"You should be able to extend the behavior of a system without having to modify that system."

"Plugin systems are the ultimate consummation, the apotheosis, of the Open-Closed Principle. They are proof positive that open-closed systems are possible, useful, and immensely powerful."
\paragraph*{Liskov Substitution Principle}
\paragraph*{Interface Segregation Principle}
\paragraph*{Interface Segregation Principle}



\subsection*{GRASP}


\subsection*{DRY}



    % Ab hier beginnt der Anhang
    %\appendix
    %\part*{\appendixname}

    %\addcontentsline{toc}{chapter}{Index}
    %\printindex

    \addcontentsline{toc}{chapter}{Literaturverzeichnis}

% Haben Sie das "biblatex"-Paket nicht installiert, benutzen Sie folgendes:
% Ohne das "biblatex"-Paket (s. bericht.sty) produziert folgendes
% "deutsche" Zitate in Literaturverzeichnissen gemaß der Norm DIN 1505,
% Teil 2 vom Jan. 1984.
% Die Zitatmarken werden alphabetisch nach Verfassern
% sortiert und sind durch abgekürzte Verfasserbuchstaben plus
% Erscheinungsjahr in eckigen Klammern gekennzeichnet.

 %\bibliographystyle{alphadin}
 %\bibliography{bericht}

%%%%%%%%%%%%%%%%%%%%%%%%%%%%%%%%%%%%%%%
% BIBLATEX
    %Benutzt man das "biblatex"-Paket, muß man folgendes schreiben:
    \def\refname{Literaturverzeichnis}
    %\bibliographystyle{apacite}
    \printbibliography

%%%%%%%%%%%%%%%%%%%%%%%%%%%%%%%%%%%%%%%

%\include{changelog}

%\newpage
%\addcontentsline{toc}{chapter}{Liste der ToDo's}
%\listoftodos[Liste der ToDo's]


\end{document}